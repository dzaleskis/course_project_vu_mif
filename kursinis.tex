\documentclass{VUMIFInfKursinis}
\usepackage{algorithmicx}
\usepackage{algorithm}
\usepackage{algpseudocode}
\usepackage{amsfonts}
\usepackage{amsmath}
\usepackage{bm}
\usepackage{color}
\usepackage{graphicx}
% \usepackage{hyperref}  % Nuorodų aktyvavimas
\usepackage{url}


% Titulinio aprašas
\university{Vilniaus universitetas}
\faculty{Matematikos ir informatikos fakultetas}
\institute{Informatikos institutas}  % Užkomentavus šią eilutę - institutas neįtraukiamas į titulinį
\department{Informatikos katedra}
\papertype{Kursinis darbas}
\title{Rikiavimo tobulinimas genetiniais algoritmais}
\titleineng{Improving sorting with genetic algorithms}
\status{3 kurso 2 grupės studentas}
\author{Deividas Zaleskis}
\supervisor{lekt. Irmantas Radavičius}
\date{Vilnius \\ \the\year}

% Nustatymai
\bibliography{bibliografija} 

\begin{document}
\maketitle

\tableofcontents

%\sectionnonum{Sąvokų apibrėžimai}
%Sutartinių ženklų, simbolių, vienetų ir terminų sutrumpinimų sąrašas (jeigu
%ženklų, simbolių, vienetų ir terminų bendras skaičius didesnis nei 10 ir
%kiekvienas iš jų tekste kartojasi daugiau nei 3 kartus).

\sectionnonum{Įvadas}
% Įvade apibūdinamas darbo tikslas, temos aktualumas ir siekiami rezultatai.

Darbo tikslas:
\textbf{pritaikyti genetinius algoritmus Šelo algoritmo tarpų sekoms generuoti}.

\bigskip

Darbo uždaviniai:

\begin{itemize}
  \item Nustatyti kriterijus tarpų sekų efektyvumui įvertinti.
  \item Paruošti aplinką eksperimentų vykdymui.
  \item Naudojant genetinius algoritmus sugeneruoti pasirinktas tarpų sekas.
  \item Atliekant eksperimentus įvertinti sugeneruotų ir pateiktų literatūroje tarpų sekų efektyvumą.
\end{itemize}

Duomenų rikiavimas yra vienas pamatinių informatikos uždavinių.
Matematiškai jis formuluojamas taip:
duotai baigtinei palyginamų elementų sekai $S = (s_1, s_2, ..., s_n)$ surasti tokį
kėlinį, kad pradinės sekos elementai būtų išdėstyti didėjančia (mažėjančia) tvarka \cite{Radavičius_Baranauskas_2013}.
Rikiavimo uždavinys yra aktualus nuo pat kompiuterių atsiradimo ir buvo laikomas vienu 
pagrindinių uždavinių, kuriuos turėtų gebėti spręsti kompiuteris \cite{10.1145/356580.356581}.
Rikiavimo uždavinio sprendimas dažnai padeda pagrindą efektyviam kito uždavinio sprendimui,
pavyzdžiui, atliekant paiešką sąraše, galima naudoti dvejetainės paieškos algoritmą tik tuo atveju,
kai sąrašas jau yra išrikiuotas.
Kadangi rikiavimo uždavinys yra fundamentalus, jam spręsti egzistuoja labai įvairių algoritmų.

Rikiavimo algoritmų yra įvairių rūšių:
paremti palyginimu (rikiuojama remiantis elementų palyginimu, o ne bendromis žiniomis apie duomenis), stabilūs (lygių elementų tvarka nekeičiama),
rikiuojantys vietoje (naudoja tik $O(1)$ papildomos atminties), etc \cite{Radavičius_Baranauskas_2013}.
Asimptotinis rikiavimo algoritmų sudėtingumas laiko atžvilgiu taip pat skiriasi:
bogo rikiavimo (angl. bogosort) \cite{10.1007/978-3-540-72914-3_17} blogiausiu atveju atliekamas palyginimų skaičius yra neribotas,
rikiavimas įterpimu (angl. insertion sort) blogiausiu atveju atlieka $O(n^2)$ palyginimų \cite{bender2006insertion},
o asimptotiškai optimalus krūvos rikiavimas (angl. heapsort) \cite{10.1145/512274.512284} atlieka $O(n\,log\,n)$ palyginimų \cite{SCHAFFER199376}.
Nepaisant rikiavimo algoritmų įvairovės, nėra algoritmo, kuris būtų geriausias visais atvejais,
kadangi praktinis efektyvumas priklauso nuo daugelio veiksnių.

Šelo rikiavimo algoritmas (angl. shellsort) \cite{10.1145/368370.368387} yra paremtas palyginimu, rikiuojantis vietoje ir nestabilus.
Šelo algoritmą galima laikyti rikiavimo įterpimu modifikacija,
kuri lygina ne gretimus, o toliau vienas nuo kito esančius elementus, taip paspartindama jų perkėlimą į galutinę poziciją.
Pagrindinė algoritmo idėja - išskaidyti rikiuojamą seką S į posekius $S_1, S_2, ..., S_n$,
kur kiekvienas posekis $S_i = (s_i, s_{i+h}, s_{i+2h}, ...)$ yra sekos S elementai, kurių pozicija skiriasi $h$.
Seka yra $h$-išrikiuota, jei išrikiuoti visi posekiai $S_i$ su tarpu $h$.


\section{Pagrindinė tiriamoji dalis}
Pagrindinėje tiriamojoje dalyje aptariama ir pagrindžiama tyrimo metodika;
pagal atitinkamas darbo dalis, nuosekliai, panaudojant lyginamosios analizės,
klasifikacijos, sisteminimo metodus bei apibendrinimus, dėstoma sukaupta ir
išanalizuota medžiaga.

\subsection{Poskyris}
Citavimo pavyzdžiai nebereikalingi.

\subsubsection{Skirsnis}
\subsubsubsection{Straipsnis}
\subsubsection{Skirsnis}
\section{Skyrius}
\subsection{Poskyris}
\subsection{Poskyris}

\sectionnonum{Išvados}
Išvadose ir pasiūlymuose, nekartojant atskirų dalių apibendrinimų,
suformuluojamos svarbiausios darbo išvados, rekomendacijos bei pasiūlymai.

\printbibliography[heading=bibintoc] % Literatūros šaltiniai aprašomi
% bibliografija.bib faile. Šaltinių sąraše nurodoma panaudota literatūra,
% kitokie šaltiniai. Abėcėlės tvarka išdėstoma tik darbe panaudotų (cituotų,
% perfrazuotų ar bent paminėtų) mokslo leidinių, kitokių publikacijų
% bibliografiniai aprašai (šiuo punktu pasirūpina LaTeX). Aprašai pateikiami
% netransliteruoti.

\appendix  % Priedai
% Prieduose gali būti pateikiama pagalbinė, ypač darbo autoriaus savarankiškai
% parengta, medžiaga. Savarankiški priedai gali būti pateikiami kompiuterio
% diskelyje ar kompaktiniame diske. Priedai taip pat vadinami ir numeruojami.
% Tekstas su priedais siejamas nuorodomis (pvz.: \ref{img:mlp}).

\section{Priedas 1}
\begin{figure}[H]
    \centering
    \includegraphics[scale=0.5]{img/MLP}
    \caption{Paveikslėlio pavyzdys}   % Antraštė įterpiama po paveikslėlio
    \label{img:mlp}
\end{figure}

\end{document}
