\documentclass{VUMIFInfKursinis}
\usepackage{algorithmicx}
\usepackage{algorithm}
\usepackage{algpseudocode}
\usepackage{amsfonts}
\usepackage{amsmath}
\usepackage{bm}
\usepackage{color}
\usepackage{graphicx}
% \usepackage{hyperref}  % Nuorodų aktyvavimas
\usepackage{url}


% Titulinio aprašas
\university{Vilniaus universitetas}
\faculty{Matematikos ir informatikos fakultetas}
\institute{Informatikos institutas}  % Užkomentavus šią eilutę - institutas neįtraukiamas į titulinį
\department{Informatikos katedra}
\papertype{Kursinis darbas}
\title{Rikiavimo tobulinimas genetiniais algoritmais}
\titleineng{Improving sorting with genetic algorithms}
\status{3 kurso 2 grupės studentas}
\author{Deividas Zaleskis}
\supervisor{lekt. Irmantas Radavičius}
\date{Vilnius \\ \the\year}

% Nustatymai
\bibliography{bibliografija} 

\begin{document}
\maketitle

\tableofcontents

%\sectionnonum{Sąvokų apibrėžimai}
%Sutartinių ženklų, simbolių, vienetų ir terminų sutrumpinimų sąrašas (jeigu
%ženklų, simbolių, vienetų ir terminų bendras skaičius didesnis nei 10 ir
%kiekvienas iš jų tekste kartojasi daugiau nei 3 kartus).

\sectionnonum{Įvadas}
% Įvade apibūdinamas darbo tikslas, temos aktualumas ir siekiami rezultatai.

Darbo tikslas:
pritaikyti genetinius algoritmus Šelo algoritmo tarpų sekoms generuoti.

\bigskip

Darbo uždaviniai:

\begin{itemize}
  \item Nustatyti kriterijus tarpų sekų efektyvumui įvertinti.
  \item Paruošti aplinką eksperimentų vykdymui.
  \item Naudojant genetinius algoritmus sugeneruoti pasirinktas tarpų sekas.
  \item Atliekant eksperimentus įvertinti sugeneruotų ir pateiktų literatūroje tarpų sekų efektyvumą.
\end{itemize}

Viena pagrindinių informatikos sąvokų yra algoritmas.
Formaliai algoritmą galima apibūdinti kaip
baigtinę seką instrukcijų, nurodančių kaip rasti nagrinėjamo uždavinio sprendinį.
Algoritmo koncepcija egzistuoja nuo senovės laikų \cite{knuth1972ancient}, tačiau atsiradus kompiuteriams,
tapo įmanoma algoritmų vykdymą automatizuoti, paverčiant juos mašininiu kodu suprantamu kompiuteriams \cite{wilkes1951preparation}.
Taip informatikos mokslas nuo teorinių šaknų \cite{turing1937computable} įgavo ir taikomąją pusę.
Beveik visus algoritmus galima suskirstyti į dvi klases: kombinatorinius algoritmus ir skaitinius algoritmus.
Skaitiniai algoritmai sprendžia tolydžius uždavinius: optimizuoti realaus argumento funkciją, išspręsti tiesinių lygčių sistemą su realiais koeficientais, etc.
Kombinatoriniai algoritmai sprendžia diskrečius uždavinius ir operuoja diskrečiais objektais: skaičiais, sąrašais, grafais, etc.
Vienas žinomiausių diskretaus uždavinio pavyzdžių yra duomenų rikiavimas.

Duomenų rikiavimas yra vienas pamatinių informatikos uždavinių.
Matematiškai jis formuluojamas taip:
duotai baigtinei palyginamų elementų sekai $S = (s_1, s_2, ..., s_n)$ pateikti tokį
kėlinį, kad pradinės sekos elementai būtų išdėstyti didėjančia (mažėjančia) tvarka \cite{Radavičius_Baranauskas_2013}.
Rikiavimo uždavinys yra aktualus nuo pat kompiuterių atsiradimo ir buvo laikomas vienu 
pagrindinių diskrečių uždavinių, kuriuos turėtų gebėti spręsti kompiuteris \cite{10.1145/356580.356581}.
Rikiavimo uždavinio sprendimas dažnai padeda pagrindą efektyviam kito uždavinio sprendimui,
pavyzdžiui, atliekant paiešką sąraše, galima taikyti dvejetainės paieškos algoritmą tik tada,
kai sąrašas yra išrikiuotas.
Kadangi rikiavimo uždavinys yra fundamentalus, jam spręsti egzistuoja labai skirtingų algoritmų.

Rikiavimo algoritmų yra įvairių:
paremtų palyginimu (elementų tvarką nustato naudojant palyginimo operatorius),
stabilių (nekeičia lygių elementų tvarkos),
rikiuojančių vietoje (nenaudoja pagalbinių duomenų struktūrų), etc.
Asimptotinis rikiavimo algoritmų sudėtingumas laiko atžvilgiu taip pat skiriasi:
bogo rikiavimo (angl. bogosort) \cite{10.1007/978-3-540-72914-3_17} blogiausiu atveju atliekamas palyginimų skaičius yra neribotas,
rikiavimas įterpimu (angl. insertion sort) blogiausiu atveju atlieka $O(n^2)$ palyginimų \cite{bender2006insertion},
o asimptotiškai optimalūs palyginimu paremti algoritmai, pavyzdžiui, krūvos rikiavimas (angl. heapsort) \cite{10.1145/512274.512284}
blogiausiu atveju atlieka $O(n\,log\,n)$ palyginimų \cite{SCHAFFER199376}.
Nepaisant rikiavimo algoritmų įvairovės, nėra algoritmo, kuris būtų geriausias visais atvejais,
nes praktinis efektyvumas priklauso nuo daugelio veiksnių.

Šelo rikiavimo algoritmas (angl. shellsort) \cite{10.1145/368370.368387} yra paremtas palyginimu, rikiuojantis vietoje ir nestabilus.
Šelo algoritmą galima laikyti rikiavimo įterpimu modifikacija,
kuri lygina ne gretimus, o toliau vienas nuo kito esančius elementus, taip paspartindama jų perkėlimą į galutinę poziciją.
Pagrindinė algoritmo idėja - išskaidyti rikiuojamą seką S į posekius $S_1, S_2, ..., S_n$,
kur kiekvienas posekis $S_i = (s_i, s_{i+h}, s_{i+2h}, ...)$ yra sekos S elementai, kurių pozicija skiriasi $h$.
Išrikiavus visus sekos $S$ posekius $S_i$ su tarpu $h$, seka tampa $h$-išrikiuota.
Remiantis tuo, jog sekai S esant $h$-išrikiuota ir ją $k$-išrikiavus, ji lieka $h$-išrikiuota \cite{GALE1972103},
galima kiekvieną algoritmo iteraciją mažinti tarpą, taip vis didinant sekos $S$ išrikiuotumą. % reiketu pasitikslinti del termino
Pritaikant šias idėjas ir rikiavimui naudojant mažėjančią tarpų seką su paskutiniu nariu $1$,
kuris garantuoja rikiavimą įterpimu paskutinėje iteracijoje,
galima užtikrinti, jog algoritmo darbo pabaigoje seka S bus pilnai išrikiuota.
Įvertinant aukščiau aprašytas mintis, nesunku pastebėti tarpų sekų įtaką Šelo rikiavimo algoritmo veikimui.

Šelo rikiavimo algoritmo efektyvumas tiesiogiai priklauso nuo pasirinktos tarpų sekos.
Weiss atlikto tyrimo \cite{weiss1991short} rezultatai rodo, jog su Sedgewick pasiūlyta seka
algoritmas veikia beveik dvigubai greičiau nei Šelo pradinis variantas, kai $n = 1000000$.
Yra įrodyta, kad šio algoritmo blogiausiu atveju atliekamų palyginimų skaičiaus apatinė riba yra
$\Omega(\frac{n\,log^2\,n}{log\,log\,n^2})$ \cite{267769}, taigi jis nėra asimptotiškai optimalus.
Tiesa, kol kas nėra rasta seka, su kuria algoritmas pasiektų šią apatinę ribą.
Kiek žinoma autoriui, asimptotiškai geriausia tarpų seka yra rasta Pratt, kuri yra formos
$2^p3^p$ ir turi $\Theta(n\,log^2\,n)$ asimptotinį sudėtingumą \cite{pratt1972shellsort},
tačiau praktikoje ji veikia lėčiau už Ciura \cite{ciura2001best} ar Tokuda \cite{10.5555/645569.659879} pasiūlytas sekas.
Daugelio praktikoje naudojamų sekų asimptotinis sudėtingumas laiko atžvilgiu lieka atvira problema,
nes jos yra rastos eksperimentiškai, tad sunku rasti matematinį modelį tinkamą jų analizei.
Vienas iš metodų, kuriuos galima taikyti efektyvių tarpų sekų radimui, yra genetinis algoritmas.

Genetinis algoritmas yra metaeuristika (metodas rasti euristikas), paremta biologijos žiniomis apie natūralios atrankos procesą.
% Genetiniai algoritmai taikomi sprendžiant paieškos ir optimizavimo uždavinius ir naudoja biologijos įkvėptus
% paveldėjimo, mutacijos, atrankos bei rekombinacijos operatorius.
% Metodo pradininkas J.H. Holland savo knygoje \cite{holland1992adaptation} apibrėžė  


 % Genetiniu algoritmu taikymas tarpu seku radimui

\section{Pagrindinė tiriamoji dalis}
Pagrindinėje tiriamojoje dalyje aptariama ir pagrindžiama tyrimo metodika;
pagal atitinkamas darbo dalis, nuosekliai, panaudojant lyginamosios analizės,
klasifikacijos, sisteminimo metodus bei apibendrinimus, dėstoma sukaupta ir
išanalizuota medžiaga.

\subsection{Poskyris}
Citavimo pavyzdžiai nebereikalingi.

\subsubsection{Skirsnis}
\subsubsubsection{Straipsnis}
\subsubsection{Skirsnis}
\section{Skyrius}
\subsection{Poskyris}
\subsection{Poskyris}

\sectionnonum{Išvados}
Išvadose ir pasiūlymuose, nekartojant atskirų dalių apibendrinimų,
suformuluojamos svarbiausios darbo išvados, rekomendacijos bei pasiūlymai.

\printbibliography[heading=bibintoc] % Literatūros šaltiniai aprašomi
% bibliografija.bib faile. Šaltinių sąraše nurodoma panaudota literatūra,
% kitokie šaltiniai. Abėcėlės tvarka išdėstoma tik darbe panaudotų (cituotų,
% perfrazuotų ar bent paminėtų) mokslo leidinių, kitokių publikacijų
% bibliografiniai aprašai (šiuo punktu pasirūpina LaTeX). Aprašai pateikiami
% netransliteruoti.

\appendix  % Priedai
% Prieduose gali būti pateikiama pagalbinė, ypač darbo autoriaus savarankiškai
% parengta, medžiaga. Savarankiški priedai gali būti pateikiami kompiuterio
% diskelyje ar kompaktiniame diske. Priedai taip pat vadinami ir numeruojami.
% Tekstas su priedais siejamas nuorodomis (pvz.: \ref{img:mlp}).

\section{Priedas 1}
\begin{figure}[H]
    \centering
    \includegraphics[scale=0.5]{img/MLP}
    \caption{Paveikslėlio pavyzdys}   % Antraštė įterpiama po paveikslėlio
    \label{img:mlp}
\end{figure}

\end{document}
