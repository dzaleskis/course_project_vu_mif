\documentclass{VUMIFInfKursinis}
\usepackage{algorithmicx}
\usepackage{algorithm}
\usepackage{algpseudocode}
\usepackage{amsfonts}
\usepackage{amsmath}
\usepackage{bm}
\usepackage{color}
\usepackage{graphicx}
\usepackage{hyperref}  % Nuorodų aktyvavimas
\usepackage{url}

\algnewcommand\algorithmicforeach{\textbf{foreach}}
\algdef{S}[FOR]{ForEach}[1]{\algorithmicforeach\ #1\ \algorithmicdo}

% Titulinio aprašas
\university{Vilniaus universitetas}
\faculty{Matematikos ir informatikos fakultetas}
\institute{Informatikos institutas}  % Užkomentavus šią eilutę - institutas neįtraukiamas į titulinį
\department{Informatikos katedra}
\papertype{Kursinis projektas}
\title{Rikiavimo tobulinimas genetiniais algoritmais}
\titleineng{Improving sorting with genetic algorithms}
\status{4 kurso 2 grupės studentas}
\author{Deividas Zaleskis}
\supervisor{Irmantas Radavičius}
\date{Vilnius \\ \the\year}

% Nustatymai
\bibliography{bibliografija} 

\begin{document}
\maketitle

\tableofcontents

\sectionnonum{Įvadas}

Čia bus įvadas (kai prisiversiu parašyt).

Darbo \textbf{tikslas}:
pritaikyti genetinius algoritmus rikiavimo algoritmų generavimui.

Darbo uždaviniai:
\begin{itemize}
  \item Atlikti Šelo algoritmo ir jo variantų literatūros analizę.
  \item Atlikti genetinių algoritmų literatūros analizę. % reiketu pataisyti, sutampa su kursinio uzdaviniu
  \item Nustatyti kriterijus rikiavimo algoritmų efektyvumui įvertinti.
  \item Paruošti aplinką rikiavimo algoritmų generavimui.
  \item Pasitelkiant genetinius algoritmus sugeneruoti rikiavimo algoritmus.
  \item Paruošti aplinką rikiavimo algoritmų tarpusavio palyginimui.
  \item Atliekant eksperimentus įvertinti sugeneruotų ir klasikinių rikiavimo algoritmų efektyvumą. % klausimas ar "klasikiniai" yra geras terminas. reikes pakeist matyt
\end{itemize}

Šis darbas sudarytas iš 7 skyrių.
Pirmame skyriuje atliekama Šelo algoritmo ir jo variantų literatūros analizė.
Antrame skyriuje atliekama genetinių algoritmų literatūros analizė. % pataisius uzdavini pakeisti ir cia
Trečiame skyriuje nustatomi kriterijai rikiavimo algoritmų efektyvumui įvertinti.
Ketvirtame skyriuje paruošiama aplinka rikiavimo algoritmų generavimui.
Penktame skyriuje pasitelkiant genetinius algoritmus generuojami rikiavimo algoritmai.
Šeštame skyriuje paruošiama aplinka rikiavimo algoritmų tarpusavio palyginimui.
Septintame skyriuje atliekant eksperimentus įvertinamas sugeneruotų ir klasikinių rikiavimo algoritmų efektyvumas. % same here

\section{Šelo algoritmas ir jo variantai}

\subsection{Šelo algoritmas}

\begin{algorithm}[H]
  \caption{Vadovėlinis Šelo rikiavimo algoritmas}\label{alg:tss}
  \begin{algorithmic}[1]
  \ForEach {$gap$ \textbf{in} $H$}
    \For {$i\gets gap+1$ \textbf{to} $N$}
      \State $j\gets i$
      \State $temp\gets S[i]$\label{alg:tss:assign1}
      \While {$j > gap$ \textbf{and} $S[j - gap] > S[j]$}\label{alg:tss:while:start}
        \State $S[j]\gets S[j - gap]$
        \State $j\gets j-gap$
      \EndWhile\label{alg:tss:while:end}
      \State $S[j]\gets temp$\label{alg:tss:assign2}
    \EndFor
  \EndFor
  \end{algorithmic}
\end{algorithm}

\subsection{Šelo algoritmo variantai}

\cite{sedgewick1996analysis}


\begin{algorithm}[H]
  \caption{Patobulintas Šelo rikiavimo algoritmas}\label{alg:iss}
  \begin{algorithmic}[1]
  \ForEach {$gap$ \textbf{in} $H$}
    \For {$i\gets gap+1$ \textbf{to} $N$}
      \If {$S[i-gap] > S[i]$}\label{alg:iss:check}
        \State $j\gets i$\label{alg:iss:inner:start}
        \State $temp\gets S[i]$
        \Repeat\label{alg:iss:loop:start}
          \State $S[j]\gets S[j - gap]$
          \State $j\gets j-gap$
        \Until {$j \le gap$ or $S[j - gap] \le S[j]$}\label{alg:iss:loop:end}
        \State $S[j]\gets temp$\label{alg:iss:inner:end}
      \EndIf
    \EndFor
  \EndFor
  \end{algorithmic}
\end{algorithm}

\cite{Radavičius_Baranauskas_2013}


% \section{Skyrius}

% Šis skyrius sudarytas iš $m$ poskyrių.
% Pirmame poskyryje nagrinėjama kažkas labai įdomaus.
% Antrame poskyryje mažiau įdomu.
% ... ir taip toliau.

% \subsection{Poskyris}

% Čia poskyris.

% \subsubsection{Poposkyris}

% Čia poposkyris.

% \subsubsection{Poposkyris}

% Ir dar vienas poposkyris.


\sectionnonum{Išvados}

Išvadose visą darbą sukišam į porą puslapių, tad čia gana svarbi dalis.
% Išvadose ir pasiūlymuose, nekartojant atskirų dalių apibendrinimų,
% suformuluojamos svarbiausios darbo išvados, rekomendacijos bei pasiūlymai.

\printbibliography[heading=bibintoc] % Literatūros šaltiniai

% \appendix  % Priedai
% Prieduose gali būti pateikiama pagalbinė, ypač darbo autoriaus savarankiškai
% parengta, medžiaga. Savarankiški priedai gali būti pateikiami kompiuterio
% diskelyje ar kompaktiniame diske. Priedai taip pat vadinami ir numeruojami.
% Tekstas su priedais siejamas nuorodomis (pvz.: \ref{img:mlp}).

\end{document}
