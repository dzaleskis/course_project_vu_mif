\documentclass{VUMIFInfKursinis}
\usepackage{algorithmicx}
\usepackage{algorithm}
\usepackage{algpseudocode}
\usepackage{amsfonts}
\usepackage{amsmath}
\usepackage{bm}
\usepackage{color}
\usepackage{graphicx}
\usepackage{hyperref}  % Nuorodų aktyvavimas
\usepackage{url}

\algnewcommand\algorithmicforeach{\textbf{foreach}}
\algdef{S}[FOR]{ForEach}[1]{\algorithmicforeach\ #1\ \algorithmicdo}

% Titulinio aprašas
\university{Vilniaus universitetas}
\faculty{Matematikos ir informatikos fakultetas}
\institute{Informatikos institutas}  % Užkomentavus šią eilutę - institutas neįtraukiamas į titulinį
\department{Informatikos katedra}
\papertype{Kursinis projektas}
\title{Rikiavimo tobulinimas genetiniais algoritmais}
\titleineng{Improving sorting with genetic algorithms}
\status{4 kurso 2 grupės studentas}
\author{Deividas Zaleskis}
\supervisor{Irmantas Radavičius}
\date{Vilnius \\ \the\year}

% Nustatymai
\bibliography{bibliografija} 

\begin{document}
\maketitle

\tableofcontents

\sectionnonum{Įvadas}

Čia bus įvadas (kai prisiversiu parašyt).

Darbo \textbf{tikslas}:
Kol kas nežinomas.


Darbo uždaviniai:
\begin{itemize}
  \item ???
  \item ??
  \item ?
  \item PROFIT!
\end{itemize}

Šis darbas sudarytas iš $n$ skyrių.
Pirmame skyriuje atliekama tas ir anas.
Antrame skyriuje vėl kažkas.
... ir taip toliau.

\section{Skyrius}

Šis skyrius sudarytas iš $m$ poskyrių.
Pirmame poskyryje nagrinėjama kažkas labai įdomaus.
Antrame poskyryje mažiau įdomu.
... ir taip toliau.

\subsection{Poskyris}

Čia poskyris.

\subsubsection{Poposkyris}

Čia poposkyris.

\subsubsection{Poposkyris}

Ir dar vienas poposkyris.

\subsection{Poskyris}

Poskyris (vėl?).

\subsubsection{Poposkyris}

Ir dar vienas poposkyris (gal jau užteks?).

\subsubsection{Poposkyris}

Ir dar vienas poposkyris (dabar jau tikrai gana).

\sectionnonum{Išvados}

Išvadose visą darbą sukišam į porą puslapių, tad čia gana svarbi dalis.
% Išvadose ir pasiūlymuose, nekartojant atskirų dalių apibendrinimų,
% suformuluojamos svarbiausios darbo išvados, rekomendacijos bei pasiūlymai.

\printbibliography[heading=bibintoc] % Literatūros šaltiniai

% \appendix  % Priedai
% Prieduose gali būti pateikiama pagalbinė, ypač darbo autoriaus savarankiškai
% parengta, medžiaga. Savarankiški priedai gali būti pateikiami kompiuterio
% diskelyje ar kompaktiniame diske. Priedai taip pat vadinami ir numeruojami.
% Tekstas su priedais siejamas nuorodomis (pvz.: \ref{img:mlp}).

\end{document}
